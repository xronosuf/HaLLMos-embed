\documentclass{ximera}

\title{H1Ex1}
\author{Jason}

\begin{document}
\begin{abstract}
    Extension Problem 1 for Homework 1
\end{abstract}
\maketitle

The extension problem list is not homework that will be collected or graded (unless you want feedback from me - in which case you should email me, or talk to me before/after class). These are problems that encompass important ideas and/or common constructions/proof-techniques used in future analytic classes (e.g. graduate courses, specialized applications in engineering, physics, etc...). I include them here for those interested in pushing a little further on the topics from this section.

This problem aims to prove one of the most commonly used tools in analysis; the \textbf{Cauchy-Schwarz Inequality};

\begin{theorem}[Cauchy-Schwarz Inequality]
    For all $x_1,x_2,y_1,y_2\in\R$,
    \[
        x_1y_1 + x_2y_2 \leq \sqrt{x_1^2 + x_2^2}\sqrt{y_1^2 + y_2^2}
    \]
\end{theorem}

There are a lot of proofs of this inequality - and most of them have some kind of a "rabbit out of a hat" moment, 
as this inequality secretly relies on a geometric justification (i.e. it is motivated by a picture, not algebra).

Perhaps the most straight forward option is to prove the following (excusing for a moment where it comes from):
\begin{hallmosEnv}[]
    For all $x_1,x_2,y_1,y_2\in\R$,
    \[
        (x_1^2 + x_2^2)(y_1^2 + y_2^2) = (x_1y_1 + x_2y_2)^2 - (x_1y_2 - x_2y_1)^2
    \]
\end{hallmosEnv}
Once you have the above proven, you can square root both sides and use the fact that, for $a\in\R$, $a^2 \geq 0$
to complete the proof of the Cauchy-Schwarz Inequality.

To motivate the Cauchy-Schwarz inequality, instead of algebra, it is better to use geometry.
In particular, the explanation and picture below is a good example (the source is from math \href{https://math.stackexchange.com/questions/3784718/geometric-interpretation-of-the-cauchy-schwarz-inequality}{stackexchange here}.)

\includegraphics{VisualizeCS.png}

Remember HaLLMos is an a.i. model that is intended to give helpful feedback on your proof, in terms of validity (structural correctness) and completeness. However, it is \textit{just} an a.i. model and may not always be correct. HaLLMos should be used to verify what you think is a finished/valid proof, but if you ever have any uncertainty or questions about your proof or the problem itself, you should definitely contact your instructor to get clarification and assistance.

\end{document}


